
% Default to the notebook output style

    


% Inherit from the specified cell style.




    
\documentclass{article}

    
    
    \usepackage{graphicx} % Used to insert images
    \usepackage{adjustbox} % Used to constrain images to a maximum size 
    \usepackage{color} % Allow colors to be defined
    \usepackage{enumerate} % Needed for markdown enumerations to work
    \usepackage{geometry} % Used to adjust the document margins
    \usepackage{amsmath} % Equations
    \usepackage{amssymb} % Equations
    \usepackage[mathletters]{ucs} % Extended unicode (utf-8) support
    \usepackage[utf8x]{inputenc} % Allow utf-8 characters in the tex document
    \usepackage{fancyvrb} % verbatim replacement that allows latex
    \usepackage{grffile} % extends the file name processing of package graphics 
                         % to support a larger range 
    % The hyperref package gives us a pdf with properly built
    % internal navigation ('pdf bookmarks' for the table of contents,
    % internal cross-reference links, web links for URLs, etc.)
    \usepackage{hyperref}
    \usepackage{longtable} % longtable support required by pandoc >1.10
    \usepackage{booktabs}  % table support for pandoc > 1.12.2
    

    
    
    \definecolor{orange}{cmyk}{0,0.4,0.8,0.2}
    \definecolor{darkorange}{rgb}{.71,0.21,0.01}
    \definecolor{darkgreen}{rgb}{.12,.54,.11}
    \definecolor{myteal}{rgb}{.26, .44, .56}
    \definecolor{gray}{gray}{0.45}
    \definecolor{lightgray}{gray}{.95}
    \definecolor{mediumgray}{gray}{.8}
    \definecolor{inputbackground}{rgb}{.95, .95, .85}
    \definecolor{outputbackground}{rgb}{.95, .95, .95}
    \definecolor{traceback}{rgb}{1, .95, .95}
    % ansi colors
    \definecolor{red}{rgb}{.6,0,0}
    \definecolor{green}{rgb}{0,.65,0}
    \definecolor{brown}{rgb}{0.6,0.6,0}
    \definecolor{blue}{rgb}{0,.145,.698}
    \definecolor{purple}{rgb}{.698,.145,.698}
    \definecolor{cyan}{rgb}{0,.698,.698}
    \definecolor{lightgray}{gray}{0.5}
    
    % bright ansi colors
    \definecolor{darkgray}{gray}{0.25}
    \definecolor{lightred}{rgb}{1.0,0.39,0.28}
    \definecolor{lightgreen}{rgb}{0.48,0.99,0.0}
    \definecolor{lightblue}{rgb}{0.53,0.81,0.92}
    \definecolor{lightpurple}{rgb}{0.87,0.63,0.87}
    \definecolor{lightcyan}{rgb}{0.5,1.0,0.83}
    
    % commands and environments needed by pandoc snippets
    % extracted from the output of `pandoc -s`
    \DefineVerbatimEnvironment{Highlighting}{Verbatim}{commandchars=\\\{\}}
    % Add ',fontsize=\small' for more characters per line
    \newenvironment{Shaded}{}{}
    \newcommand{\KeywordTok}[1]{\textcolor[rgb]{0.00,0.44,0.13}{\textbf{{#1}}}}
    \newcommand{\DataTypeTok}[1]{\textcolor[rgb]{0.56,0.13,0.00}{{#1}}}
    \newcommand{\DecValTok}[1]{\textcolor[rgb]{0.25,0.63,0.44}{{#1}}}
    \newcommand{\BaseNTok}[1]{\textcolor[rgb]{0.25,0.63,0.44}{{#1}}}
    \newcommand{\FloatTok}[1]{\textcolor[rgb]{0.25,0.63,0.44}{{#1}}}
    \newcommand{\CharTok}[1]{\textcolor[rgb]{0.25,0.44,0.63}{{#1}}}
    \newcommand{\StringTok}[1]{\textcolor[rgb]{0.25,0.44,0.63}{{#1}}}
    \newcommand{\CommentTok}[1]{\textcolor[rgb]{0.38,0.63,0.69}{\textit{{#1}}}}
    \newcommand{\OtherTok}[1]{\textcolor[rgb]{0.00,0.44,0.13}{{#1}}}
    \newcommand{\AlertTok}[1]{\textcolor[rgb]{1.00,0.00,0.00}{\textbf{{#1}}}}
    \newcommand{\FunctionTok}[1]{\textcolor[rgb]{0.02,0.16,0.49}{{#1}}}
    \newcommand{\RegionMarkerTok}[1]{{#1}}
    \newcommand{\ErrorTok}[1]{\textcolor[rgb]{1.00,0.00,0.00}{\textbf{{#1}}}}
    \newcommand{\NormalTok}[1]{{#1}}
    
    % Define a nice break command that doesn't care if a line doesn't already
    % exist.
    \def\br{\hspace*{\fill} \\* }
    % Math Jax compatability definitions
    \def\gt{>}
    \def\lt{<}
    % Document parameters
    \title{LNMeanReversionModel}
    
    
    

    % Pygments definitions
    
\makeatletter
\def\PY@reset{\let\PY@it=\relax \let\PY@bf=\relax%
    \let\PY@ul=\relax \let\PY@tc=\relax%
    \let\PY@bc=\relax \let\PY@ff=\relax}
\def\PY@tok#1{\csname PY@tok@#1\endcsname}
\def\PY@toks#1+{\ifx\relax#1\empty\else%
    \PY@tok{#1}\expandafter\PY@toks\fi}
\def\PY@do#1{\PY@bc{\PY@tc{\PY@ul{%
    \PY@it{\PY@bf{\PY@ff{#1}}}}}}}
\def\PY#1#2{\PY@reset\PY@toks#1+\relax+\PY@do{#2}}

\expandafter\def\csname PY@tok@gd\endcsname{\def\PY@tc##1{\textcolor[rgb]{0.63,0.00,0.00}{##1}}}
\expandafter\def\csname PY@tok@gu\endcsname{\let\PY@bf=\textbf\def\PY@tc##1{\textcolor[rgb]{0.50,0.00,0.50}{##1}}}
\expandafter\def\csname PY@tok@gt\endcsname{\def\PY@tc##1{\textcolor[rgb]{0.00,0.27,0.87}{##1}}}
\expandafter\def\csname PY@tok@gs\endcsname{\let\PY@bf=\textbf}
\expandafter\def\csname PY@tok@gr\endcsname{\def\PY@tc##1{\textcolor[rgb]{1.00,0.00,0.00}{##1}}}
\expandafter\def\csname PY@tok@cm\endcsname{\let\PY@it=\textit\def\PY@tc##1{\textcolor[rgb]{0.25,0.50,0.50}{##1}}}
\expandafter\def\csname PY@tok@vg\endcsname{\def\PY@tc##1{\textcolor[rgb]{0.10,0.09,0.49}{##1}}}
\expandafter\def\csname PY@tok@m\endcsname{\def\PY@tc##1{\textcolor[rgb]{0.40,0.40,0.40}{##1}}}
\expandafter\def\csname PY@tok@mh\endcsname{\def\PY@tc##1{\textcolor[rgb]{0.40,0.40,0.40}{##1}}}
\expandafter\def\csname PY@tok@go\endcsname{\def\PY@tc##1{\textcolor[rgb]{0.53,0.53,0.53}{##1}}}
\expandafter\def\csname PY@tok@ge\endcsname{\let\PY@it=\textit}
\expandafter\def\csname PY@tok@vc\endcsname{\def\PY@tc##1{\textcolor[rgb]{0.10,0.09,0.49}{##1}}}
\expandafter\def\csname PY@tok@il\endcsname{\def\PY@tc##1{\textcolor[rgb]{0.40,0.40,0.40}{##1}}}
\expandafter\def\csname PY@tok@cs\endcsname{\let\PY@it=\textit\def\PY@tc##1{\textcolor[rgb]{0.25,0.50,0.50}{##1}}}
\expandafter\def\csname PY@tok@cp\endcsname{\def\PY@tc##1{\textcolor[rgb]{0.74,0.48,0.00}{##1}}}
\expandafter\def\csname PY@tok@gi\endcsname{\def\PY@tc##1{\textcolor[rgb]{0.00,0.63,0.00}{##1}}}
\expandafter\def\csname PY@tok@gh\endcsname{\let\PY@bf=\textbf\def\PY@tc##1{\textcolor[rgb]{0.00,0.00,0.50}{##1}}}
\expandafter\def\csname PY@tok@ni\endcsname{\let\PY@bf=\textbf\def\PY@tc##1{\textcolor[rgb]{0.60,0.60,0.60}{##1}}}
\expandafter\def\csname PY@tok@nl\endcsname{\def\PY@tc##1{\textcolor[rgb]{0.63,0.63,0.00}{##1}}}
\expandafter\def\csname PY@tok@nn\endcsname{\let\PY@bf=\textbf\def\PY@tc##1{\textcolor[rgb]{0.00,0.00,1.00}{##1}}}
\expandafter\def\csname PY@tok@no\endcsname{\def\PY@tc##1{\textcolor[rgb]{0.53,0.00,0.00}{##1}}}
\expandafter\def\csname PY@tok@na\endcsname{\def\PY@tc##1{\textcolor[rgb]{0.49,0.56,0.16}{##1}}}
\expandafter\def\csname PY@tok@nb\endcsname{\def\PY@tc##1{\textcolor[rgb]{0.00,0.50,0.00}{##1}}}
\expandafter\def\csname PY@tok@nc\endcsname{\let\PY@bf=\textbf\def\PY@tc##1{\textcolor[rgb]{0.00,0.00,1.00}{##1}}}
\expandafter\def\csname PY@tok@nd\endcsname{\def\PY@tc##1{\textcolor[rgb]{0.67,0.13,1.00}{##1}}}
\expandafter\def\csname PY@tok@ne\endcsname{\let\PY@bf=\textbf\def\PY@tc##1{\textcolor[rgb]{0.82,0.25,0.23}{##1}}}
\expandafter\def\csname PY@tok@nf\endcsname{\def\PY@tc##1{\textcolor[rgb]{0.00,0.00,1.00}{##1}}}
\expandafter\def\csname PY@tok@si\endcsname{\let\PY@bf=\textbf\def\PY@tc##1{\textcolor[rgb]{0.73,0.40,0.53}{##1}}}
\expandafter\def\csname PY@tok@s2\endcsname{\def\PY@tc##1{\textcolor[rgb]{0.73,0.13,0.13}{##1}}}
\expandafter\def\csname PY@tok@vi\endcsname{\def\PY@tc##1{\textcolor[rgb]{0.10,0.09,0.49}{##1}}}
\expandafter\def\csname PY@tok@nt\endcsname{\let\PY@bf=\textbf\def\PY@tc##1{\textcolor[rgb]{0.00,0.50,0.00}{##1}}}
\expandafter\def\csname PY@tok@nv\endcsname{\def\PY@tc##1{\textcolor[rgb]{0.10,0.09,0.49}{##1}}}
\expandafter\def\csname PY@tok@s1\endcsname{\def\PY@tc##1{\textcolor[rgb]{0.73,0.13,0.13}{##1}}}
\expandafter\def\csname PY@tok@sh\endcsname{\def\PY@tc##1{\textcolor[rgb]{0.73,0.13,0.13}{##1}}}
\expandafter\def\csname PY@tok@sc\endcsname{\def\PY@tc##1{\textcolor[rgb]{0.73,0.13,0.13}{##1}}}
\expandafter\def\csname PY@tok@sx\endcsname{\def\PY@tc##1{\textcolor[rgb]{0.00,0.50,0.00}{##1}}}
\expandafter\def\csname PY@tok@bp\endcsname{\def\PY@tc##1{\textcolor[rgb]{0.00,0.50,0.00}{##1}}}
\expandafter\def\csname PY@tok@c1\endcsname{\let\PY@it=\textit\def\PY@tc##1{\textcolor[rgb]{0.25,0.50,0.50}{##1}}}
\expandafter\def\csname PY@tok@kc\endcsname{\let\PY@bf=\textbf\def\PY@tc##1{\textcolor[rgb]{0.00,0.50,0.00}{##1}}}
\expandafter\def\csname PY@tok@c\endcsname{\let\PY@it=\textit\def\PY@tc##1{\textcolor[rgb]{0.25,0.50,0.50}{##1}}}
\expandafter\def\csname PY@tok@mf\endcsname{\def\PY@tc##1{\textcolor[rgb]{0.40,0.40,0.40}{##1}}}
\expandafter\def\csname PY@tok@err\endcsname{\def\PY@bc##1{\setlength{\fboxsep}{0pt}\fcolorbox[rgb]{1.00,0.00,0.00}{1,1,1}{\strut ##1}}}
\expandafter\def\csname PY@tok@kd\endcsname{\let\PY@bf=\textbf\def\PY@tc##1{\textcolor[rgb]{0.00,0.50,0.00}{##1}}}
\expandafter\def\csname PY@tok@ss\endcsname{\def\PY@tc##1{\textcolor[rgb]{0.10,0.09,0.49}{##1}}}
\expandafter\def\csname PY@tok@sr\endcsname{\def\PY@tc##1{\textcolor[rgb]{0.73,0.40,0.53}{##1}}}
\expandafter\def\csname PY@tok@mo\endcsname{\def\PY@tc##1{\textcolor[rgb]{0.40,0.40,0.40}{##1}}}
\expandafter\def\csname PY@tok@kn\endcsname{\let\PY@bf=\textbf\def\PY@tc##1{\textcolor[rgb]{0.00,0.50,0.00}{##1}}}
\expandafter\def\csname PY@tok@mi\endcsname{\def\PY@tc##1{\textcolor[rgb]{0.40,0.40,0.40}{##1}}}
\expandafter\def\csname PY@tok@gp\endcsname{\let\PY@bf=\textbf\def\PY@tc##1{\textcolor[rgb]{0.00,0.00,0.50}{##1}}}
\expandafter\def\csname PY@tok@o\endcsname{\def\PY@tc##1{\textcolor[rgb]{0.40,0.40,0.40}{##1}}}
\expandafter\def\csname PY@tok@kr\endcsname{\let\PY@bf=\textbf\def\PY@tc##1{\textcolor[rgb]{0.00,0.50,0.00}{##1}}}
\expandafter\def\csname PY@tok@s\endcsname{\def\PY@tc##1{\textcolor[rgb]{0.73,0.13,0.13}{##1}}}
\expandafter\def\csname PY@tok@kp\endcsname{\def\PY@tc##1{\textcolor[rgb]{0.00,0.50,0.00}{##1}}}
\expandafter\def\csname PY@tok@w\endcsname{\def\PY@tc##1{\textcolor[rgb]{0.73,0.73,0.73}{##1}}}
\expandafter\def\csname PY@tok@kt\endcsname{\def\PY@tc##1{\textcolor[rgb]{0.69,0.00,0.25}{##1}}}
\expandafter\def\csname PY@tok@ow\endcsname{\let\PY@bf=\textbf\def\PY@tc##1{\textcolor[rgb]{0.67,0.13,1.00}{##1}}}
\expandafter\def\csname PY@tok@sb\endcsname{\def\PY@tc##1{\textcolor[rgb]{0.73,0.13,0.13}{##1}}}
\expandafter\def\csname PY@tok@k\endcsname{\let\PY@bf=\textbf\def\PY@tc##1{\textcolor[rgb]{0.00,0.50,0.00}{##1}}}
\expandafter\def\csname PY@tok@se\endcsname{\let\PY@bf=\textbf\def\PY@tc##1{\textcolor[rgb]{0.73,0.40,0.13}{##1}}}
\expandafter\def\csname PY@tok@sd\endcsname{\let\PY@it=\textit\def\PY@tc##1{\textcolor[rgb]{0.73,0.13,0.13}{##1}}}

\def\PYZbs{\char`\\}
\def\PYZus{\char`\_}
\def\PYZob{\char`\{}
\def\PYZcb{\char`\}}
\def\PYZca{\char`\^}
\def\PYZam{\char`\&}
\def\PYZlt{\char`\<}
\def\PYZgt{\char`\>}
\def\PYZsh{\char`\#}
\def\PYZpc{\char`\%}
\def\PYZdl{\char`\$}
\def\PYZhy{\char`\-}
\def\PYZsq{\char`\'}
\def\PYZdq{\char`\"}
\def\PYZti{\char`\~}
% for compatibility with earlier versions
\def\PYZat{@}
\def\PYZlb{[}
\def\PYZrb{]}
\makeatother


    % Exact colors from NB
    \definecolor{incolor}{rgb}{0.0, 0.0, 0.5}
    \definecolor{outcolor}{rgb}{0.545, 0.0, 0.0}



    
    % Prevent overflowing lines due to hard-to-break entities
    \sloppy 
    % Setup hyperref package
    \hypersetup{
      breaklinks=true,  % so long urls are correctly broken across lines
      colorlinks=true,
      urlcolor=blue,
      linkcolor=darkorange,
      citecolor=darkgreen,
      }
    % Slightly bigger margins than the latex defaults
    
    \geometry{verbose,tmargin=1in,bmargin=1in,lmargin=1in,rmargin=1in}
    
    

    \begin{document}
    
    
    \maketitle
    
    

    

    \section{Fitting FX Rates Data to Log-normal Mean Reversion Model}


    I have come across Mean Reverting process in Quant Finance context from
time to time, I am taking this chance to write down some collectives,
and learn a bit of writting in \textbf{iPython Notebook}.


    \subsection{The Maths}


    When it comes to Quantitative model, it always start with a bit of
maths, here we go\ldots{}


    \subsubsection{Starting with The Common Mean Reverting Process}


    The SDE for a mean reverting process looks like:
\[\mathrm{d}x=\alpha(\theta-x) \mathrm{d}t + \sigma \mathrm{d}W\] where:
- $\alpha$ is the mean reversion speed, - $\theta$ is the mean reversion
level, - $\sigma$ is the volatility, - $W$ is a Brownian Motion. To
solve for $x$ analytically, we assume \[z=e^{\alpha t} x\] Apply Ito's
lemma \[
\begin{equation}
\begin{split}
\mathrm{d}z & =\alpha e^{\alpha t} x \mathrm{d}t + e^{\alpha t} \mathrm{d}x  \\
 & = \alpha e^{\alpha t} x \mathrm{d}t + \alpha e^{\alpha t}(\theta-x) \mathrm{d}t + e^{\alpha t} \sigma \mathrm{d}W \\
 & = \alpha \theta e^{\alpha t} \mathrm{d}t + e^{\alpha t} \sigma \mathrm{d}W \\
\end{split}
\end{equation}
\] Integrate, get:
\[z_T = z_0 + \theta (e^{\alpha T}-1) + \int_0^T \sigma e^{\alpha t} \mathrm{d}W_t\]
So:
\[x_T=e^{-\alpha T}x_0 + \theta (1-e^{-\alpha T}) + e^{-\alpha T} \int_0^T \sigma e^{\alpha T} \mathrm{d}W_t\]

Intuition: The process is called ``mean reverting'' because there is a
force (the strength of which described as mean reversion speed $\alpha$,
generally positive) dragging it back to the ``mean'' level ($\theta$).
Because:
\[\lim_{T \to 0} e^{-\alpha T} = 1 \mathrm{, and } \lim_{T \to +\infty} e^{-\alpha T}=0\]
Looking at the analytical form solution $x_T$, the first term ``dial
down'' to zero as $T$ increases; while the second term ``dial up'' from
$0$ to $\theta$ as $T$ increases. The mean of the process tend to the
mean level from initial level $x_0$. The third term represents
dispersion, $x_T$ is normally distributed.
\[\lim_{T \to +\infty} \mathbb{E}[x] = \theta\] and,
\[\lim_{T \to +\infty} \mathbb{Var}[x] = \frac{\sigma^2}{2\alpha}\]
Important fact here is the variance does not increase linear on time.

The well-known Hull-White Interest Rate short rate model is based on the
above process, the enhancement was replacing $\theta$ with a time
function which allows calibrating to the initial yield curve term
structure.


    \subsubsection{Log-normal Mean Reverting Process}


    With what we derived in the previous section, let $x=\ln{y}$, we call
$y$ follows a log-normal mean reverting process. $y=e^x$, and it is
log-normally distributed.


    \subsection{Data Fitting}


    A model is only useful if it could be fitted to data and tested.


    \subsubsection{Linear Regression}


    Obviously the difference between the log-normal mean reverting process
and the non-log-normal version is just a matter of taking the $log$
value, which means we just need to work out either of them to make it
useful. And just at the first glance we can tell the non-log-normal
version is nicer to work with. Ok, enough rubbish.. Writing
differentiation in discrete form:
\[\mathrm{d}x=x_{i+1}-x_i\]and\[\mathrm{d}t=\Delta t\]and\[\mathrm{d}W=\epsilon\sim\mathbb{N}(0,\sqrt{\Delta t})\]
We get: \[
\begin{equation}
\begin{split}
x_{i+1}-x_i & =\alpha (\theta-x_i) \Delta t + \sigma \epsilon \\
x_{i+1} & = \alpha \theta \Delta t + (1-\alpha\Delta t)x_i + \sigma\epsilon
\end{split}
\end{equation}
\] What it says is $x_{i+1}$ is a linear function of $x_i$ plus a white
noise term. So if we plot $x_{i+1}$ against $x_i$, we expect the
scatters to stay more or less a straight line, kind of giving a first
impression we've chosen a good model for the data.

    Using the linear polynomial form for fitting \[f(x)=P_1 x + P_2\] We
have \[
\begin{equation}
\begin{split}
p_2&=\alpha \theta \Delta t \\
p_1&=(1-\alpha\Delta t)
\end{split}
\end{equation}
\] And work out to: \[
\begin{equation}
\begin{split}
\alpha &=\frac{1-P_1}{\Delta t}\\
\mathrm{and} \\
\theta &= \frac{P_2}{\frac{\alpha}{\Delta t}}\\
\end{split}
\end{equation}
\] particularly, \[\mathrm{MeanLevel}=e^{\theta}\] which suggests a
``mean level'' the FX rate reverts to. The following \emph{Python} code
generates the figure showing this level and the historical observations.


    \subsubsection{Python}


    I used Python code to try fitting the above model to USD-GBP historical
rates (2009-Mar-24 to 2014-Mar-24).

    \begin{Verbatim}[commandchars=\\\{\}]
{\color{incolor}In [{\color{incolor}1}]:} \PY{k+kn}{import} \PY{n+nn}{matplotlib.pyplot} \PY{k+kn}{as} \PY{n+nn}{plt}
        \PY{k+kn}{import} \PY{n+nn}{numpy} \PY{k+kn}{as} \PY{n+nn}{np}
        \PY{k+kn}{from} \PY{n+nn}{datetime} \PY{k+kn}{import} \PY{n}{datetime}
        \PY{k+kn}{from} \PY{n+nn}{scipy.stats} \PY{k+kn}{import} \PY{n}{norm}
        \PY{k+kn}{import} \PY{n+nn}{scipy.stats} \PY{k+kn}{as} \PY{n+nn}{stats}
        
        \PY{c}{\PYZsh{}import data from CSV}
        
        \PY{n}{convertfunc} \PY{o}{=} \PY{k}{lambda} \PY{n}{x}\PY{p}{:} \PY{n}{datetime}\PY{o}{.}\PY{n}{strptime}\PY{p}{(}\PY{n}{x}\PY{p}{,} \PY{l+s}{\PYZsq{}}\PY{l+s+si}{\PYZpc{}d}\PY{l+s}{/}\PY{l+s}{\PYZpc{}}\PY{l+s}{m/}\PY{l+s}{\PYZpc{}}\PY{l+s}{Y}\PY{l+s}{\PYZsq{}}\PY{p}{)}\PY{p}{;}
        \PY{n}{col\PYZus{}headers} \PY{o}{=} \PY{p}{[}\PY{l+s}{\PYZdq{}}\PY{l+s}{Date}\PY{l+s}{\PYZdq{}}\PY{p}{,} \PY{l+s}{\PYZdq{}}\PY{l+s}{USD\PYZus{}GBP}\PY{l+s}{\PYZdq{}}\PY{p}{,} \PY{l+s}{\PYZdq{}}\PY{l+s}{USD\PYZus{}EUR}\PY{l+s}{\PYZdq{}}\PY{p}{,} \PY{l+s}{\PYZdq{}}\PY{l+s}{USD\PYZus{}JPY}\PY{l+s}{\PYZdq{}}\PY{p}{,} \PY{l+s}{\PYZdq{}}\PY{l+s}{USDCHF}\PY{l+s}{\PYZdq{}}\PY{p}{]}\PY{p}{;}
        
                                
        \PY{n}{RawRate} \PY{o}{=} \PY{n}{np}\PY{o}{.}\PY{n}{genfromtxt}\PY{p}{(}\PY{l+s}{\PYZsq{}}\PY{l+s}{/Users/zipenghuang/Dropbox/data/FXRates.csv}\PY{l+s}{\PYZsq{}}\PY{p}{,} \PY{n}{delimiter}\PY{o}{=}\PY{l+s}{\PYZsq{}}\PY{l+s}{,}\PY{l+s}{\PYZsq{}}\PY{p}{,}
                                \PY{c}{\PYZsh{}dtype=\PYZsq{}object\PYZsq{} , }
                                \PY{n}{skip\PYZus{}header}\PY{o}{=}\PY{l+m+mi}{1}\PY{p}{,}
                                \PY{n}{usecols} \PY{o}{=} \PY{p}{(}\PY{l+m+mi}{1}\PY{p}{,}\PY{l+m+mi}{2}\PY{p}{,}\PY{l+m+mi}{3}\PY{p}{,}\PY{l+m+mi}{4}\PY{p}{)}\PY{p}{,}
                                \PY{n}{names}\PY{o}{=}\PY{n}{col\PYZus{}headers}\PY{p}{,}\PY{n}{converters}\PY{o}{=}\PY{p}{\PYZob{}}\PY{l+s}{\PYZdq{}}\PY{l+s}{Date}\PY{l+s}{\PYZdq{}}\PY{p}{:} \PY{n}{convertfunc}\PY{p}{\PYZcb{}}\PY{p}{)}\PY{p}{;}
        
        \PY{n}{RawDate} \PY{o}{=} \PY{n}{np}\PY{o}{.}\PY{n}{genfromtxt}\PY{p}{(}\PY{l+s}{\PYZsq{}}\PY{l+s}{/Users/zipenghuang/Dropbox/data/FXRates.csv}\PY{l+s}{\PYZsq{}}\PY{p}{,} \PY{n}{delimiter}\PY{o}{=}\PY{l+s}{\PYZsq{}}\PY{l+s}{,}\PY{l+s}{\PYZsq{}}\PY{p}{,}
                                \PY{n}{dtype}\PY{o}{=}\PY{l+s}{\PYZsq{}}\PY{l+s}{object}\PY{l+s}{\PYZsq{}}\PY{p}{,} 
                                \PY{n}{skip\PYZus{}header}\PY{o}{=}\PY{l+m+mi}{1}\PY{p}{,}
                                \PY{n}{usecols} \PY{o}{=} \PY{l+m+mi}{0}\PY{p}{,}
                                \PY{n}{names}\PY{o}{=}\PY{n}{col\PYZus{}headers}\PY{p}{,}\PY{n}{converters}\PY{o}{=}\PY{p}{\PYZob{}}\PY{l+s}{\PYZdq{}}\PY{l+s}{Date}\PY{l+s}{\PYZdq{}}\PY{p}{:} \PY{n}{convertfunc}\PY{p}{\PYZcb{}}\PY{p}{)}\PY{p}{;}
                                
        
        \PY{c}{\PYZsh{}data fitting}
                         
        \PY{n}{IX\PYZus{}DateSort} \PY{o}{=} \PY{n}{RawDate}\PY{o}{.}\PY{n}{argsort}\PY{p}{(}\PY{p}{)}\PY{p}{;}
        
        \PY{n}{Rate} \PY{o}{=} \PY{n}{RawRate}\PY{p}{[}\PY{l+s}{\PYZsq{}}\PY{l+s}{USD\PYZus{}GBP}\PY{l+s}{\PYZsq{}}\PY{p}{]}\PY{p}{[}\PY{n}{IX\PYZus{}DateSort}\PY{p}{]}\PY{p}{;}
        \PY{n}{LogRate} \PY{o}{=} \PY{n}{np}\PY{o}{.}\PY{n}{log}\PY{p}{(}\PY{n}{Rate}\PY{p}{)}\PY{p}{;}
        
        \PY{n}{x} \PY{o}{=} \PY{n}{LogRate}\PY{p}{[}\PY{l+m+mi}{0}\PY{p}{:}\PY{o}{\PYZhy{}}\PY{l+m+mi}{1}\PY{p}{]}\PY{p}{;}
        \PY{n}{y} \PY{o}{=} \PY{n}{LogRate}\PY{p}{[}\PY{l+m+mi}{1}\PY{p}{:}\PY{p}{]}\PY{p}{;}
        
        \PY{n}{P} \PY{o}{=} \PY{n}{np}\PY{o}{.}\PY{n}{polyfit}\PY{p}{(}\PY{n}{x}\PY{p}{,}\PY{n}{y}\PY{p}{,}\PY{l+m+mi}{1}\PY{p}{)}\PY{p}{;}
        
        \PY{n}{alpha} \PY{o}{=} \PY{p}{(}\PY{l+m+mf}{1.0}\PY{o}{\PYZhy{}}\PY{n}{P}\PY{p}{[}\PY{l+m+mi}{0}\PY{p}{]}\PY{p}{)}\PY{o}{/}\PY{p}{(}\PY{l+m+mf}{1.0}\PY{o}{/}\PY{l+m+mi}{365}\PY{p}{)}\PY{p}{;}
        \PY{n}{theta} \PY{o}{=} \PY{n}{P}\PY{p}{[}\PY{l+m+mi}{1}\PY{p}{]}\PY{o}{/}\PY{n}{alpha}\PY{o}{/}\PY{p}{(}\PY{l+m+mf}{1.0}\PY{o}{/}\PY{l+m+mi}{365}\PY{p}{)}\PY{p}{;}
        
        \PY{n}{X} \PY{o}{=} \PY{n}{np}\PY{o}{.}\PY{n}{array}\PY{p}{(}\PY{p}{[}\PY{n}{x}\PY{o}{*}\PY{o}{*}\PY{l+m+mi}{1}\PY{p}{,} \PY{n}{x}\PY{o}{*}\PY{o}{*}\PY{l+m+mi}{0}\PY{p}{]}\PY{p}{)}\PY{p}{;}
        \PY{n}{Residuals} \PY{o}{=} \PY{n}{y} \PY{o}{\PYZhy{}} \PY{n}{np}\PY{o}{.}\PY{n}{dot}\PY{p}{(}\PY{n}{P}\PY{p}{,} \PY{n}{X}\PY{p}{)}\PY{p}{;}
        
        
        \PY{c}{\PYZsh{}calculate mean and std of the residuals}
        \PY{n}{mu}\PY{p}{,} \PY{n}{std} \PY{o}{=} \PY{n}{norm}\PY{o}{.}\PY{n}{fit}\PY{p}{(}\PY{n}{Residuals}\PY{p}{)}\PY{p}{;}
        
        \PY{n}{plt}\PY{o}{.}\PY{n}{figure}\PY{p}{(}\PY{p}{)}\PY{p}{;}
        \PY{c}{\PYZsh{} Plot the histogram.}
        \PY{n}{plt}\PY{o}{.}\PY{n}{hist}\PY{p}{(}\PY{n}{Residuals}\PY{p}{,} \PY{n}{bins}\PY{o}{=}\PY{l+m+mi}{100}\PY{p}{,} \PY{n}{normed}\PY{o}{=}\PY{n+nb+bp}{True}\PY{p}{,} \PY{n}{alpha}\PY{o}{=}\PY{l+m+mf}{0.6}\PY{p}{,} \PY{n}{color}\PY{o}{=}\PY{l+s}{\PYZsq{}}\PY{l+s}{g}\PY{l+s}{\PYZsq{}}\PY{p}{)}\PY{p}{;}
        \PY{c}{\PYZsh{} Plot the PDF.}
        \PY{n}{xmin}\PY{p}{,} \PY{n}{xmax} \PY{o}{=} \PY{n}{plt}\PY{o}{.}\PY{n}{xlim}\PY{p}{(}\PY{p}{)}\PY{p}{;}
        \PY{n}{x} \PY{o}{=} \PY{n}{np}\PY{o}{.}\PY{n}{linspace}\PY{p}{(}\PY{n}{xmin}\PY{p}{,} \PY{n}{xmax}\PY{p}{,} \PY{l+m+mi}{100}\PY{p}{)}\PY{p}{;}
        \PY{n}{p} \PY{o}{=} \PY{n}{norm}\PY{o}{.}\PY{n}{pdf}\PY{p}{(}\PY{n}{x}\PY{p}{,} \PY{n}{mu}\PY{p}{,} \PY{n}{std}\PY{p}{)}\PY{p}{;}
        \PY{n}{plt}\PY{o}{.}\PY{n}{plot}\PY{p}{(}\PY{n}{x}\PY{p}{,} \PY{n}{p}\PY{p}{,} \PY{l+s}{\PYZsq{}}\PY{l+s}{k}\PY{l+s}{\PYZsq{}}\PY{p}{,} \PY{n}{linewidth}\PY{o}{=}\PY{l+m+mi}{2}\PY{p}{)}\PY{p}{;}
        \PY{n}{plt}\PY{o}{.}\PY{n}{title}\PY{p}{(}\PY{l+s}{\PYZdq{}}\PY{l+s}{Fit results: mu = }\PY{l+s+si}{\PYZpc{}.8f}\PY{l+s}{,  std = }\PY{l+s+si}{\PYZpc{}.8f}\PY{l+s}{\PYZdq{}} \PY{o}{\PYZpc{}} \PY{p}{(}\PY{n}{mu}\PY{p}{,} \PY{n}{std}\PY{p}{)}\PY{p}{)}\PY{p}{;}
        \PY{n}{plt}\PY{o}{.}\PY{n}{show}\PY{p}{(}\PY{p}{)}\PY{p}{;}
        
        \PY{n}{plt}\PY{o}{.}\PY{n}{figure}\PY{p}{(}\PY{p}{)}\PY{p}{;}
        \PY{n}{stats}\PY{o}{.}\PY{n}{probplot}\PY{p}{(}\PY{n}{Residuals}\PY{p}{,} \PY{n}{dist}\PY{o}{=}\PY{l+s}{\PYZdq{}}\PY{l+s}{norm}\PY{l+s}{\PYZdq{}}\PY{p}{,} \PY{n}{plot}\PY{o}{=}\PY{n}{plt}\PY{p}{)}\PY{p}{;}
        \PY{n}{plt}\PY{o}{.}\PY{n}{title}\PY{p}{(}\PY{l+s}{\PYZdq{}}\PY{l+s}{Q\PYZhy{}Q Plot}\PY{l+s}{\PYZdq{}}\PY{p}{)}\PY{p}{;}
        \PY{n}{plt}\PY{o}{.}\PY{n}{show}\PY{p}{(}\PY{p}{)}\PY{p}{;}
        
        
        \PY{c}{\PYZsh{}Plot time\PYZhy{}series against the mean level calculated from the fit}
        \PY{n}{MeanLevel} \PY{o}{=} \PY{n}{np}\PY{o}{.}\PY{n}{exp}\PY{p}{(}\PY{n}{theta}\PY{p}{)}\PY{p}{;}
        
        \PY{n}{plt}\PY{o}{.}\PY{n}{figure}\PY{p}{(}\PY{p}{)}
        \PY{n}{plt}\PY{o}{.}\PY{n}{plot}\PY{p}{(}\PY{n}{RawDate}\PY{p}{[}\PY{n}{IX\PYZus{}DateSort}\PY{p}{]}\PY{p}{,}\PY{n}{RawRate}\PY{p}{[}\PY{l+s}{\PYZsq{}}\PY{l+s}{USD\PYZus{}GBP}\PY{l+s}{\PYZsq{}}\PY{p}{]}\PY{p}{[}\PY{n}{IX\PYZus{}DateSort}\PY{p}{]}\PY{p}{,}\PY{l+s}{\PYZsq{}}\PY{l+s}{b\PYZhy{}}\PY{l+s}{\PYZsq{}}\PY{p}{)}
        \PY{n}{plt}\PY{o}{.}\PY{n}{xlabel}\PY{p}{(}\PY{l+s}{\PYZsq{}}\PY{l+s}{Date}\PY{l+s}{\PYZsq{}}\PY{p}{)}
        \PY{n}{plt}\PY{o}{.}\PY{n}{ylabel}\PY{p}{(}\PY{l+s}{\PYZsq{}}\PY{l+s}{USD\PYZhy{}GBP}\PY{l+s}{\PYZsq{}}\PY{p}{)}
        \PY{n}{plt}\PY{o}{.}\PY{n}{title}\PY{p}{(}\PY{l+s}{\PYZsq{}}\PY{l+s}{USD\PYZhy{}GBP Time Series}\PY{l+s}{\PYZsq{}}\PY{p}{)}
        \PY{n}{plt}\PY{o}{.}\PY{n}{axhline}\PY{p}{(}\PY{n}{y}\PY{o}{=}\PY{n}{MeanLevel} \PY{p}{,}\PY{n}{linewidth}\PY{o}{=}\PY{l+m+mi}{1}\PY{p}{,} \PY{n}{color}\PY{o}{=}\PY{l+s}{\PYZsq{}}\PY{l+s}{r}\PY{l+s}{\PYZsq{}}\PY{p}{)}
        \PY{n}{plt}\PY{o}{.}\PY{n}{show}\PY{p}{(}\PY{p}{)}
\end{Verbatim}

    The fit result is: \[f(x) =  0.99196289 x -0.00375743
\] The residuals vs.~the fitted normal distribution looks like this.

    \begin{Verbatim}[commandchars=\\\{\}]
{\color{incolor}In [{\color{incolor}2}]:} \PY{k+kn}{from} \PY{n+nn}{IPython.display} \PY{k+kn}{import} \PY{n}{Image}
        \PY{n}{Image}\PY{p}{(}\PY{n}{filename}\PY{o}{=}\PY{l+s}{\PYZsq{}}\PY{l+s}{/Users/zipenghuang/Dropbox/writing/iPythonNotebook/LNMeanReversionModel/Res.png}\PY{l+s}{\PYZsq{}}\PY{p}{)}
\end{Verbatim}
\texttt{\color{outcolor}Out[{\color{outcolor}2}]:}
    
    \begin{center}
    \adjustimage{max size={0.9\linewidth}{0.9\paperheight}}{LNMeanReversionModel_files/LNMeanReversionModel_17_0.png}
    \end{center}
    { \hspace*{\fill} \\}
    

    Does not look very normal, also confirmed by Q-Q plot below:

    \begin{Verbatim}[commandchars=\\\{\}]
{\color{incolor}In [{\color{incolor}3}]:} \PY{k+kn}{from} \PY{n+nn}{IPython.display} \PY{k+kn}{import} \PY{n}{Image}
        \PY{n}{Image}\PY{p}{(}\PY{n}{filename}\PY{o}{=}\PY{l+s}{\PYZsq{}}\PY{l+s}{/Users/zipenghuang/Dropbox/writing/iPythonNotebook/LNMeanReversionModel/QQ.png}\PY{l+s}{\PYZsq{}}\PY{p}{)}
\end{Verbatim}
\texttt{\color{outcolor}Out[{\color{outcolor}3}]:}
    
    \begin{center}
    \adjustimage{max size={0.9\linewidth}{0.9\paperheight}}{LNMeanReversionModel_files/LNMeanReversionModel_19_0.png}
    \end{center}
    { \hspace*{\fill} \\}
    

    The fit suggests a mean reverting level of 0.62656021383357741, below is
this level plotted against the time series of the rates.

    \begin{Verbatim}[commandchars=\\\{\}]
{\color{incolor}In [{\color{incolor}4}]:} \PY{k+kn}{from} \PY{n+nn}{IPython.display} \PY{k+kn}{import} \PY{n}{Image}
        \PY{n}{Image}\PY{p}{(}\PY{n}{filename}\PY{o}{=}\PY{l+s}{\PYZsq{}}\PY{l+s}{/Users/zipenghuang/Dropbox/writing/iPythonNotebook/LNMeanReversionModel/TS.png}\PY{l+s}{\PYZsq{}}\PY{p}{)}
\end{Verbatim}
\texttt{\color{outcolor}Out[{\color{outcolor}4}]:}
    
    \begin{center}
    \adjustimage{max size={0.9\linewidth}{0.9\paperheight}}{LNMeanReversionModel_files/LNMeanReversionModel_21_0.png}
    \end{center}
    { \hspace*{\fill} \\}
    


    \subsubsection{Matlab}


    Below is basically a quick play with Matlab trying to do the same thing
done with Python.

    \begin{Verbatim}[commandchars=\\\{\}]
{\color{incolor}In [{\color{incolor}3}]:} \PY{k+kn}{from} \PY{n+nn}{IPython.display} \PY{k+kn}{import} \PY{n}{Image}
        \PY{n}{Image}\PY{p}{(}\PY{n}{filename}\PY{o}{=}\PY{l+s}{\PYZsq{}}\PY{l+s}{/Users/zipenghuang/Dropbox/writing/iPythonNotebook/LNMeanReversionModel/FXfit.png}\PY{l+s}{\PYZsq{}}\PY{p}{)}
\end{Verbatim}
\texttt{\color{outcolor}Out[{\color{outcolor}3}]:}
    
    \begin{center}
    \adjustimage{max size={0.9\linewidth}{0.9\paperheight}}{LNMeanReversionModel_files/LNMeanReversionModel_24_0.png}
    \end{center}
    { \hspace*{\fill} \\}
    

    And a quick distribution fit of the residuals to a normal distribution
does not seem to suggest the model choice was excellent..

    \begin{Verbatim}[commandchars=\\\{\}]
{\color{incolor}In [{\color{incolor}1}]:} \PY{k+kn}{from} \PY{n+nn}{IPython.display} \PY{k+kn}{import} \PY{n}{Image}
        \PY{n}{Image}\PY{p}{(}\PY{n}{filename}\PY{o}{=}\PY{l+s}{\PYZsq{}}\PY{l+s}{/Users/zipenghuang/Dropbox/writing/iPythonNotebook/LNMeanReversionModel/ResidualDistributionFit.png}\PY{l+s}{\PYZsq{}}\PY{p}{)}
\end{Verbatim}
\texttt{\color{outcolor}Out[{\color{outcolor}1}]:}
    
    \begin{center}
    \adjustimage{max size={0.9\linewidth}{0.9\paperheight}}{LNMeanReversionModel_files/LNMeanReversionModel_26_0.png}
    \end{center}
    { \hspace*{\fill} \\}
    

    Anyway, Matlab gives the linear fit results as:

                Linear model Poly1:
        f(x) = p1*x + p2
Coefficients (with 95% confidence bounds):
        p1 =      0.9916  (0.9862, 0.997)
        p2 =   -0.003796  (-0.006311, -0.001281)

Goodness of fit:
  SSE: 0.01834
  R-square: 0.9896
  Adjusted R-square: 0.9896
  RMSE: 0.003675
                

    % Add a bibliography block to the postdoc
    
    
    
    \end{document}
